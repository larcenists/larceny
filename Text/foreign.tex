% -*- LaTeX -*-

\documentstyle[10pt]{article}
\newcommand{\reg}[1]{{\sc \%#1}}

\topmargin      -2.0cm
\oddsidemargin   0.0cm
\evensidemargin  0.0cm
\textwidth      6.5in
\textheight     9.5in
\parindent       0.0cm
\parskip         0.4cm

\title{Larceny Note \#19: \\
       Foreign Procedure Interface \\
       {\tenrm (proposal)}}
\author{Lars Thomas Hansen}

\begin{document}
\maketitle

We describe a small number of extra primops which, with the necessary
Scheme-level glue, facilitates calling procedures written in other
languages. For each implementation, there is a set of defined languages
which Scheme knows about, and the definitions below are cast in terms
of some unknown language ``X''. For each language there is a defined set
of valid return types. The definitions below sometimes refer to a return
type in terms of the variable ``Y''.

The implementation of these primitives will vary between implementations
of Larceny; however, a suggested implementation for the Sparc version with
foreign C procedures is described below.

\begin{description}

\item[\verb+(foreign-X-procedure-address name)  =>  integer+]

{\tt Foreign-X-procedure-address} takes the name of the foreign procedure
as it would appear in a program in the foreign language, and returns an
exact integer which is the core address of the foreign procedure. This
assumes that the foreign address will not move; an implementation can
get around moving foreign procedure by having a level of indirection.

\item[\verb+(foreign-X-procedure-call-returning-Y integer args) => object+]

{\tt Foreign-X-procedure-call-returning-Y} takes an integer (the address of
the foreign procedure) and the list of arguments to that procedure, and calls
that procedure. The effects of passing an invalid address are undefined.
The type of the returned value is given by Y; the returned object is a
Scheme object.

\end{description}

For the C language:

\begin{verbatim}
        foreign-C-procedure-address
        foreign-C-procedure-call-returning-char
        foreign-C-procedure-call-returning-int32
        foreign-C-procedure-call-returning-uint32
        foreign-C-procedure-call-returning-ieee32
        foreign-C-procedure-call-returning-ieee64
        foreign-C-procedure-call-returning-string
        foreign-C-procedure-call-returning-taggedptr
\end{verbatim}

The type ``string'' indicates that the procedure returns a pointer
to a null-terminated array of characters. A copy of the string is made
in the Scheme heap and returned.

The type ``taggedptr'' indicates that the procedure returns a tagged word
which lives in the Scheme heap. This object could have been passed to the
C procedure or it could have been allocated in the Scheme heap by the C
procedure; the implementation of the latter is out of the scope of this
document.

In principle we could arrange for user-defined return types like structures
(but not necessarily unions). These would be mapped from the C structure to
an array or a list. However, the runtime then needs detailed knowledge of
how the structures are laid out by the compiler.

On the other hand, it is difficult to pass Scheme structures and expect the
C code to fill them in correctly.

One can malloc foreign memory, but this requires that we have an access
facility; one can be written using C calls, but that is pretty slow.

One possibility is to provide for returning C arrays [which are then copied
into Scheme arrays], for some scalar types.

A higher-level wrapper for these low-level facilities:

\begin{verbatim}
(define (foreign-C-function name argument-types result-type)
  (let ((addr (foreign-C-procedure-address name)))
    (if (not name)
        (error "Undefined foreign procedure " name)
        (let ((caller 
                (case result-type
                  ((char)      foreign-C-procedure-call-returning-char)
		  ((int32)     foreign-C-procedure-call-returning-int32)
                  ((uint32)    foreign-C-procedure-call-returning-uint32)
                  ((ieee32)    foreign-C-procedure-call-returning-ieee32)
                  ((ieee64)    foreign-C-procedure-call-returning-ieee64)
                  ((string)    foreign-C-procedure-call-returning-string)
                  ((taggedptr) foreign-C-procedure-call-returning-taggedptr)
                  (else (error ...)))))
          (lambda args
            (map (lambda (a b)
                   (case a
                     ((char) (if (not (character? b))
                                 (error ...)))
                     ((int32) ...)
                     ...))
                 argument-types
                 args)
            (caller addr args))))))
\end{verbatim}

	
\end{document}
