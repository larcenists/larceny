% -*- LaTeX -*-

\documentstyle[10pt]{article}
\newcommand{\reg}[1]{{\sc \%#1}}

\topmargin      -2.0cm
\oddsidemargin   0.0cm
\evensidemargin  0.0cm
\textwidth      6.5in
\textheight     9.5in
\parindent       0.0cm
\parskip         0.4cm

\title{Larceny Note \#23: System Invariants}
\author{Lars Thomas Hansen}

\begin{document}
\maketitle

\begin{abstract}
This document attempts to list all invariants in the implementation of Larceny.
Each invariant is given an identifying number (which does {\em not} change
as this document evolves), the idea being that a comment in the source can
refer to an invariant by its number in order to justify the code. Each
invariant mentioned in this document is probably documented in some other
note, and if so, a reference to the pertinent note is found herein.

When a piece of code is written or debugged and its correctness depends on
a perceived invariant, that invariant must be located and understood before
the code can be considered correct.
\end{abstract}

\section{Definitions}

Execution is always\footnote{At least for the time being.} in one of three
{\em modes}. In {\em C mode} the Scheme register contents are all saved
in the software registers in the globals table, and the calling conventions
in effect are those of the C compiler. In {\em Scheme mode}, some Scheme
registers are mapped to hardware registers, and all of the defined mappings
are in effect. In {\em Millicode mode}, we are mostly in Scheme mode,
but may sometimes use the temporary registers for internal parameter passing,
and we may temporarily switch out of Scheme mode by doing a {\tt save}
instruction to get more registers or to prepare to go into C mode.


\section{Invariants}

\begin{enumerate}
% 1
\item
When executing in Scheme mode, and except after executing a RETURN
instruction and before executing a RESTORE instruction, REG0 is a pointer to
the currently executing procedure. This is required by the MacScheme calling
conventions and mentioned parenthetically in [Larceny Note \#1].

% 2
\item
When entering a millicode procedure, REG0 always points to the currently
executing procedure, and {\tt \%o7} is used to save the return address.
The first argument is passed in RESULT, the second in ARGREG2, and the third
in ARGREG2. [Larceny Note \#1.]

% 3
\item
RESULT, ARGREG2, and ARGREG3 are preserved across calls to memory management
millicode routines (although if a result is returned, RESULT is of course
destroyed). [Larceny Note \#3.]

\end{enumerate}

\end{document}

