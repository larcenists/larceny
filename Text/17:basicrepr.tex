% -*- TeX -*-
%
% Ultimate definition of the basic representations in Larceny.
% Derived representations are not shown here.

% -*- LaTeX -*-

\documentstyle[10pt]{article}
\newcommand{\reg}[1]{{\sc \%#1}}

\topmargin      -2.0cm
\oddsidemargin   0.0cm
\evensidemargin  0.0cm
\textwidth      6.5in
\textheight     9.5in
\parindent       0.0cm
\parskip         0.4cm

\title{Larceny Note \#2001: \\
       Basic Representations}
\author{Lars Thomas Hansen}

\begin{document}
\maketitle

A pointer has a 2 or 3 bit type tag in the low order bits. Structures on
the heap (except pairs) have a type tag in the first word. All heap
structures must be aligned on an 8-byte boundary.

The following are the tagged pointer formats.

\begin{minipage}{\linewidth}
\begin{verbatim}
  xxxx xxxx  xxxx xxxx  xxxx xxxx  xxxx xx00   fixnum
  xxxx xxxx  xxxx xxxx  xxxx xxxx  xxxx xx10   immediate
  pppp pppp  pppp pppp  pppp pppp  pppp p001   pointer to pair 
  pppp pppp  pppp pppp  pppp pppp  pppp p011   pointer to vector struct
  pppp pppp  pppp pppp  pppp pppp  pppp p101   pointer to bytevector struct
  pppp pppp  pppp pppp  pppp pppp  pppp p111   pointer to procedure struct
\end{verbatim}
\end{minipage}

The following are the immediate formats.

\begin{minipage}{\linewidth}
\begin{verbatim}
  0000 0000  0000 0000  0000 0000  0000 0010   #f
  0000 0000  0000 0000  0000 0000  0000 0110   #t
  0000 0000  0000 0000  0000 0000  0000 1010   empty list
  xxxx xxxx  xxxx xxxx  xxxx xxxx  0001 0110   miscellaneous
  0000 0000  cccc cccc  0000 0000  0010 0110   character
  0sss ssss  ssss ssss  ssss ssss  100x xx10   reserved header
  0sss ssss  ssss ssss  ssss ssss  101x xx10   vector-like header
  0sss ssss  ssss ssss  ssss ssss  110x xx10   bytevector-like header
  0sss ssss  ssss ssss  ssss ssss  1111 1110   procedure header
\end{verbatim}
\end{minipage}

The "xxx" bits in the low byte of a header can be used by the mutator to
distinguish between different subtypes of structures; they are neither
used nor examined by the collector.

The "s" bits must contain the size of the data structure in bytes, not
including the header word, and not including padding (see below). The
collector will correctly round "bytevector" and "reserved" lengths up
to a word boundary. Hence, for these structures, the "s" bits give the
correct length of the datum, like the length of a string. For vectors
and procedures the s field must be evenly divisible by 4 (low two bits
are 0).

Since all pointers are doubleword-aligned, some vectors, bytevectors,
and procedures will have to be padded out to an even number of words.
The mutator must allocate the extra word but may choose to leave it
uninitialized; the collector will take the size of the vector into
account and skip the padding. If the word is initialized, it may or
may not be copied along with the structure during a collection.

The structure layouts are these:

\begin{itemize}
\item
Pairs: A pair has two words, the car (low word) and the cdr (high word).
The pair pointer points to the car of the pair.

\item
A vector-like structure has the header word in the low position, followed
by tagged pointers in all locations of the vector. The vector pointer
points to the header word.

\item
A bytevector-like structure has the header word in the low position,
followed by untagged bytes in all locations of the vector. The bytevector
pointer points to the header word.

\item
A procedure structure has the header word in the low position, followed by
tagged pointers in all locations of the procedure structure. To the
collector, and with the exception of the pointer and header tags, a
procedure looks just like a vector-like structure. The interpretation of
each word of the procedure structure is up to the mutator and is not
documented here. The procedure pointer points to the header word.

\item
There is a "reserved" header but currently no "reserved" data layout; the
collector does not know about the "reserved" header.
\end{itemize}

\end{document}
