% -*- LaTeX -*-

\documentstyle[10pt]{article}
\newcommand{\reg}[1]{{\sc \%#1}}

\topmargin      -2.0cm
\oddsidemargin   0.0cm
\evensidemargin  0.0cm
\textwidth      6.5in
\textheight     9.5in
\parindent       0.0cm
\parskip         0.4cm

\title{Larceny Note \#21: \\
       The Larceny Development Environment}
\author{R\'{e}my Evard}

\begin{document}
\maketitle

\section{Introduction}

The Larceny development environment is a collection of directories
populated with files related to the Larceny project.  Most are used
to build either the larceny executable or the larceny heap, while others
are documentation.  This document describes how to work in this
environment.

\section{The Environment}

As there are multiple people working on the project, some form of file
synchronization must be used.  This is done using 
RCS, the Revision Control System. There is a central directory of RCS files 
which contains the real files.
Each person working on the project has their own mirror of this hierarchy.
To read a file, they ``check out'' the file, to write the file, they ``lock''
it, and when they have completed it, they ``check in'' the file.  Only
files which have been checked in are available to other people on the
project. This minimizes confusion caused by having multiple authors.

Each person has their own larceny tree - the root of that tree is where
most of the work is done, and thus has a special name: $\sim$larceny.
The larceny executable, the heap, and most of the utilities are located
in $\sim$larceny.

The actual directories and files are subject to frequent change.
Each directory contains a file called {\tt Index}, which describes 
each file in that directory.  The Index in $\sim$larceny also
describes each directory in the project.

\section{Getting Started}

To create a larceny directory tree, first create the root directory,
which will be your $\sim$larceny.  Currently, the conventional name
for this directory is {\tt /research/owyhee/$<$user$>$/scheme313}.
Make this directory your current working directory (i.e. cd into it).
Now run the script {\tt make-larceny-tree}.  Since you don't currently
have a copy of this program, you will need to run someone else's.  
This will create the directory tree for you, and fill in the RCS links.
Finally, run the script {\tt checkout}.  Again, you will need
to run someone else's copy (or check it out directly first).  This
script will populate your directory with the files that are part of
the larceny development environment.

The steps to do this are:
\begin{verbatim}
  % cd /research/owyhee/<user>
  % mkdir scheme313
  % cd scheme313
  % ../../lth/scheme313/make-larceny-tree
  % co checkout
  % checkout
\end{verbatim}

\section{Building larceny}

To make the larceny executable, make sure you are in $\sim$larceny,
then type {\tt make}.   This will compile and assemble the correct
files (assuming you have them checked out), and produce a file called
{\tt larceny} in $\sim$larceny.

\section{Building the heap}

To build the larceny heap, make sure you are iin $\sim$larceny, 
then type {\tt build}.  This will start a Chez Scheme process running
the make utilties.  It will load several packages, then end at
the Chez prompt.

To create the heap, say {\tt (make-larceny-heap)}.  It will compile
and assemble the necessary files, then load each file that is part
of the runtime environment, and then create a file called larceny.heap, 
which will be located in $\sim$larceny.

It is possible to create heaps in other ways - Larceny Note \#5 details
some of them.

\section{Active editing}

Initially, all the files that you checked out will be read-only - you 
will not be able to edit them.  This is part of using RCS.  Only 
the source files are read-only.  The files which you produce: larceny,
.o files, and so on. are all readable and writeable by you.

Once you decide you would like to edit a file, you need to check out the
file and lock it.  This keeps other people from editing the file while
you are working on it.  (It is possible to use some forms of revision
software which allow multiple people to edit the same file, but that
hasn't really been needed as yet.)  To lock a file, you say 
{\tt co -l $<$file$>$}.  

Once you have completed editing it and you would like to make the new
version available o other people, you check it in.  This is done by
saying {\tt ci $<$file$>$}.

If you have checked out a lot of files (a common occurence), you can
check them all in again by going to $\sim$larceny and running the
{\tt checkin} script.

If other people have checked in a lot of files (another common
occurence), and you would like to get the newest versions of all those
files, you can run the {\tt checkout} script.  Again, this must be
run while in $\sim$larceny.

\section{Creating a new file}

If you create a new file to add to the Larceny environment, you must
do an initial check-in.  This is done by saying {\tt ci $<$file$>$}.
RCS will ask you for a description of the file - type whatever you'd
like.  

At this point, you should add an entry to the {\tt Index} in the current
directory, describing a bit about the file.  You will have to lock the
Index to edit it. 

You may also want to consider whether or not you have introduced some
new concepts which should be mentioned in the Glarceny (Larceny Note #\666}.

\section{Adding a primop}
Follow these instructions:
\begin{enumerate}
\item Find Lars.
\item Take him to Fall Street Cafe.
\item Mention that you would like to add a primop.
\item Take careful notes, and enjoy thoroughly.
\item Add the primop.
\item Get Lars to fix whatever broke...
\end{enumerate}

\end{document}
