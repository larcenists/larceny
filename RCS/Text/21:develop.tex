% -*- LaTeX -*-

\documentstyle[10pt]{article}
\newcommand{\reg}[1]{{\sc \%#1}}

\topmargin      -2.0cm
\oddsidemargin   0.0cm
\evensidemargin  0.0cm
\textwidth      6.5in
\textheight     9.5in
\parindent       0.0cm
\parskip         0.4cm

\title{Larceny Note \#21: \\
       The Larceny Development Environment}
\author{R\'{e}my Evard and Lars Thomas Hansen}

\begin{document}
\maketitle

\section{Introduction}

The Larceny development environment is a collection of directories
populated with files related to the Larceny project.  Most are used
to build either the larceny executable or the larceny heap, while others
are documentation.  This document describes how to work in this
environment.

\section{The Environment}

As there are multiple people working on the project, some form of file
synchronization must be used.  This is done using RCS, the Revision Control
System. There is a central directory of RCS files which contains the real
files.  Each person working on the project has their own mirror of this
hierarchy.  To read a file, they ``check out'' the file, to write the file,
they ``lock'' it, and when they have completed it, they ``check in'' the
file.  Only files which have been checked in are available to other people
on the project. This minimizes confusion caused by having multiple authors.

Each person has their own Larceny tree - the root of that tree is where
most of the work is done, and thus has a special name: $\sim$larceny.
The larceny executable, the heap, and most of the utilities are located
in $\sim$larceny.

The actual directories and files are subject to frequent change.
Each directory contains a file called {\tt Index}, which describes 
each file in that directory.  The Index in $\sim$larceny also
describes each directory in the project.

\section{Getting Started}

To create a larceny directory tree, first create the root directory,
which will be your $\sim$larceny.  Currently, the conventional name
for this directory is {\tt /research/owyhee/$<$user$>$/scheme313}.
Make this directory your current working directory (i.e. cd into it).
Now run the script {\tt make-larceny-tree}.  Since you don't currently
have a copy of this program, you will need to run someone else's.  
This will create the directory tree for you, and fill in the RCS links.
Finally, run the script {\tt checkout}.  Again, you will need
to run someone else's copy (or check it out directly first).  This
script will populate your directory with the files that are part of
the larceny development environment.

The steps to do this are:
\begin{verbatim}
  % cd /research/owyhee/<user>
  % mkdir scheme313
  % cd scheme313
  % /research/owyhee/scheme313/make-larceny-tree
  % co checkout
  % checkout
\end{verbatim}

\section{Building larceny}

To make the larceny executable, make sure you are in $\sim$larceny,
then type {\tt make}.   This will compile and assemble the correct
files (assuming you have them checked out), and produce a file called
{\tt larceny} in $\sim$larceny.

\section{Building the heap}

To build the larceny heap, make sure you are iin $\sim$larceny, 
then type {\tt build}.  This will start a Chez Scheme process running
the make utilties.  It will load several packages, then end at
the Chez prompt.

To create the heap, say {\tt (make-larceny-heap)}.  It will compile
and assemble the necessary files, then load each file that is part
of the runtime environment, and then create a file called larceny.heap, 
which will be located in $\sim$larceny.

It is possible to create heaps in other ways - Larceny Note \#5 details
some of them.

\section{Active editing}

Initially, all the files that you checked out will be read-only - you 
will not be able to edit them.  This is part of using RCS.  Only 
the source files are read-only.  The files which you produce: larceny,
.o files, and so on. are all readable and writeable by you.

Once you decide you would like to edit a file, you need to check out the
file and lock it.  This keeps other people from editing the file while
you are working on it.  (It is possible to use some forms of revision
software which allow multiple people to edit the same file, but that
hasn't really been needed as yet.)  To lock a file, you say 
{\tt co -l $<$file$>$}.  

Once you have completed editing it and you would like to make the new
version available o other people, you check it in.  This is done by
saying {\tt ci $<$file$>$}.

If you have checked out a lot of files (a common occurence), you can
check them all in again by going to $\sim$larceny and running the
{\tt checkin} script.

If other people have checked in a lot of files (another common
occurence), and you would like to get the newest versions of all those
files, you can run the {\tt checkout} script.  Again, this must be
run while in $\sim$larceny.

\section{Creating a new file}

If you create a new file to add to the Larceny environment, you must
do an initial check-in.  This is done by saying {\tt ci $<$file$>$}.
RCS will ask you for a description of the file - type whatever you'd
like.  

At this point, you should add an entry to the {\tt Index} in the current
directory, describing a bit about the file.  You will have to lock the
Index to edit it. 

You may also want to consider whether or not you have introduced some
new concepts which should be mentioned in the Glarceny (Larceny Note \#666).

\section{Adding a primop}

The stuff that Larceny is made of...

\subsection{The story according to R\'{e}my}

Follow these instructions:
\begin{enumerate}
\item Find Lars.
\item Take him to the Fall Creek Bakery.
\item Mention that you would like to add a primop.
\item Take careful notes, and enjoy thoroughly.
\item Add the primop.
\item Get Lars to fix whatever broke...
\end{enumerate}

\subsection{The story according to Lars}

Adding a primop can be a very educational experience. I'll try to expand on
steps 3 and 4 above, as these are the vital ones (although a brownie and a
cup of coffee helps).

For the time being, a primop is defined to the compiler by adding it to the
table \verb+$usual-integrable-procedures$+ in {\tt Compiler/pass1.imp.sch}.
This table is a list of 5-element lists. The first element is the name of
the primop as it will appear in a Scheme source file.  The second element is
the number of arguments the primop takes. The third argument is the name of
the primop as it should appear in the MacScheme assembly language output
from the compiler; this is the name that the assembler recognizes. The
fourth element is a predicate whose function is a little unclear to me right
now but which does something with literal arguments to the primop. The fifth
element is the MacScheme machine opcode for the primop; for Larceny-specific
primops it is always -1, and in fact, in Larceny this element is always
ignored.

So, to add a primop, you first define an entry for the primop in this table.
If the primop has an immediate form, you must also add an entry to the
other table in this file, \verb+$immediate-primops$+. This table is a list
of lists where the sublist has a primop name (source form, I think) and the
MacScheme opcode.

When the primop has been defined in this table, you usually want to define
a stub procedure for it so that it can be passed around as an ``ordinary''
procedure. The stubs are defined in {\tt Lib/integrable-procs.scm} and are
all of the form

\begin{verbatim}
          (define foo (lambda (x) (foo x)))
\end{verbatim}

for a primop {\tt foo} of one argument. Note that the MIT-style definitions
should *not* be used here. The assumption is that this procedure will be
compiled with optimization turned on, and on such a setting, the compiler
will distinguish between the two forms. The Indiana-style definition above
will not be self-referential, but an MIT-style definition will be. So watch
it!

The next thing to do is to write the code generation stub for the primop. In
the current system this means adding a primop to the primop list in
{\tt Sparc/gen-primops.sch}. This file contains an association list which
associates a primop name with a procedure. The procedure is the code generation
stub for the primop. Such a stub takes at least one argument (the assembler
structure) and then additional arguments depending on the primop. For
example, a two-argument primop {\tt bar} would have a definition much like
this (in the definition of the list):

\begin{minipage}{\linewidth}
\begin{verbatim}
          (cons 'bar
                (lambda (as r)
                  ...))
\end{verbatim}
\end{minipage}

where {\tt r} is the register name in which the second argument to {\tt bar}
is passed (the first is always in {\tt RESULT}, of course).

The process of creating the code generation stub is actually a little more
complicated, since there may be more than one stub for each primop: one
for immediate args, one for the general case. The current system punts that
by always moving immediates into regs and then doing the general case, but
this is clearly the wrong thing.

A rewrite of {\tt Sparc/gen-primops.sch} which is much cleaner has been
done, but is untested and is not currently in use.

In principle you have now added the primop to the system. However, some
primops need support beyond what can be done in-line, and the code
generation stub will in these cases generate a branch thru the millicode
table to a millicode procedure. If this is the case, there are more steps:

First, you have to create a slot in the millicode table for the millicode
procedure. First, edit the file {\tt millicode.cfg} to define the slot number
for the new slot. This is simple; just add one to the end of the file and
then update the count up at the top of the file (try it, it's obvious).
Then, edit the file {\tt Sparc/tables.s} to add another slot at the end.
Each slot is an unconditional branch to the millicode procedure followed
by a no-op instruction.

The millicode table is now in good shape. At this point, you will need to
write the SPARC assembly code for the millicode operation itself. A millicode
procedure goes into one of three files: {\tt Sparc/memory.s} if it has to
do with memory management; {\tt Sparc/generic.s} if it has to do with generic
arithmetic; and in {\tt Sparc/glue.s} otherwise.

Of course, the millicode procedure may call C code to do some stuff for it;
C code which supports millicode goes in {\tt Sys/cglue.c} or 
{\tt Sys/memsupport.c}, depending on what it does: the latter is for memory
management stuff; the former for everything else.

Remember that after {\em changing} a primop, you must recompile everything
that uses the primop, or you will be in deep trouble. A wholesale recompile
is sometimes the best. Often you can get away with reassembling.

\end{document}
