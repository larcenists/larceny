% -*- TeX -*-
% $Id: numrepr.txt,v 1.1 91/12/06 14:59:03 lth Exp Locker: lth $

% -*- LaTeX -*-

\documentstyle[10pt]{article}
\newcommand{\reg}[1]{{\sc \%#1}}

\topmargin      -2.0cm
\oddsidemargin   0.0cm
\evensidemargin  0.0cm
\textwidth      6.5in
\textheight     9.5in
\parindent       0.0cm
\parskip         0.4cm


\title{Larceny Note \#4: \\
       Number Representations}
\author{Lars Thomas Hansen}

\begin{document}
\maketitle

Fixnums are unboxed and kept in the high 30 bits of a word, with the
two low bits always 0 (figure 1).

\begin{minipage}{\linewidth}
\begin{verbatim}
                          +------------------------------+--+
                          |       fixnum                 |00|
                          +------------------------------+--+
\end{verbatim}
\centerline{Figure 1: Fixnum}
\end{minipage}

Bignums are bytevector-like with the sign in the first two bytes
(\#x0000 for 0 or positive, \#xFFFF for negative), followed by a digit
count (two bytes) and then base-$2^{32}$ digits in the next words.
with the least significant word first; each word is stored big-endian
(figure 2).

\begin{minipage}{\linewidth}
\begin{verbatim}
                          +------------------------+--------+
                          |       length           | header |
                          +------------------------+--------+
                          | sign          |   digitcount    |
                          +---------------------------------+
                          |              lsd                |
                          +---------------------------------+
                          ...
\end{verbatim}
\centerline{Figure 2: Bignum with 32-bit digits}
\end{minipage}

Each word is stored big-endian, and can be interpreted as two 16-bit
digits, also in big-endian fashion within the word; interpreted this
way, the bignum gets a funny access pattern (figure 3).
The digit count is still the number of 32-bit digits used.

\begin{minipage}{\linewidth}
\begin{verbatim}
                          +------------------------+--------+
                          |       length           | header |
                          +------------------------+--------+
                          | sign          |   digitcount    |
                          +---------------------------------+
                          |    nlsd       |      lsd        |
                          +---------------------------------+
                          ...
\end{verbatim}
\centerline{Figure 3: Bignum with 16-bit digits}
\end{minipage}

Ratnums (figure 4) are vector-like, with the first word of
the vector being the numerator as a scheme object (fixnum or bignum),
and the second word being the denominator (greater than 1).

\begin{minipage}{\linewidth}
\begin{verbatim}
                          +------------------------+--------+
                          |       vectorlength     | header |
                          +------------------------+--------+
                          |       numerator                 |
                          +---------------------------------+
                          |       denominator               |
                          +---------------------------------+
\end{verbatim}
\centerline{Figure 4: Ratnum}
\end{minipage}

Rectnums (figure 5) look like ratnums, except that the
first word is the real part (an integer or ratnum) and the second word
is the imaginary part (ditto). Both parts are exact reals, and the
imaginary part is nonzero.

\begin{minipage}{\linewidth}
\begin{verbatim}
                          +------------------------+--------+
                          |       vectorlength     | header |
                          +------------------------+--------+
                          |       real-part                 |
                          +---------------------------------+
                          |       imag-part                 |
                          +---------------------------------+
\end{verbatim}
\centerline{Figure 5: Rectnum}
\end{minipage}

Flonums (IEEE double) are bytevector-like. The first word is unused,
and the two next words contain the double. The rationale for the unused
word is this: since everything in the system is aligned on an 8-byte
boundary, one word will be wasted anyway. By putting it before the
number rather than after, the number becomes 8-byte aligned, and we can use
a ``load double'' instruction to load it. (Figure 6.)

\begin{minipage}{\linewidth}
\begin{verbatim}
                          +------------------------+--------+
                          |      length            | header |
                          +------------------------+--------+
                          |      unused                     |
                          +---------------------------------+
                          |      IEEE double precision      |
                          |                                 |
                          +---------------------------------+
\end{verbatim}
\centerline{Figure 6: Flonum}
\end{minipage}

Compnums (two IEEE doubles) are bytevector-like. The first word is
unused (see the description of the flonum for a rationale).  The two
next words contain the real part. The two last words contain the
imaginary part. (Figure 7.)

\begin{minipage}{\linewidth}
\begin{verbatim}
                          +------------------------+--------+
                          |      length            | hdrtag |
                          +------------------------+--------+
                          |      unused                     |
                          +---------------------------------+
                          |      IEEE double precision      |
                          |      (real part)                |
                          +---------------------------------+
                          |      IEEE double precision      |
                          |      (imaginary part)           |
                          +---------------------------------+
\end{verbatim}
\centerline{Figure 7: Compnum}
\end{minipage}

\end{document}
