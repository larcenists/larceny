% -*-TeX -*-
% $Id$

% -*- LaTeX -*-

\documentstyle[10pt]{article}
\newcommand{\reg}[1]{{\sc \%#1}}

\topmargin      -2.0cm
\oddsidemargin   0.0cm
\evensidemargin  0.0cm
\textwidth      6.5in
\textheight     9.5in
\parindent       0.0cm
\parskip         0.4cm


\title{Larceny Note \#6: \\
       File Formats}
\author{Lars Thomas Hansen}

\begin{document}
\maketitle

\begin{abstract}
This document describes the formats of ``.lap'', ``.mal'', ``.lop'' and
``.heap'' files in detail.
\end{abstract}

\section{Segment Format}

A segment is a pair:

\begin{itemize}
\item
The car of the segment is a byte vector (externally represented as a
vector in some host implementations) containing the raw bytes of the
SPARC object code. This code is fully compiled and will not need to be
further patched at load time. The bytes are laid out in order of increasing
addresses for fast loading. The byte vector is exactly long enough.

\item
The cdr of the segment is a vector containing all non-inlined constants
used by the procedure. Constants may share structure if the assembler
determined that they could. Examples of constants include lists, flonums, 
and code and constant vectors for nested procedures which are to be created
at run time. Each entry in the constant vector is a pair. The car of the
pair is the tag of the entry; the cdr is the actual data. There are five
kinds of tags: ``data'', ``codevector'', ``constantvector'', ``global'',
and ``bits''. 

\end{itemize}

The tags and their data are interpreted thusly:

\begin{itemize}
\item
The ``data'' tag is used for raw data which is to be dumped
into the heap as is, but with the correct tags added. For example,
\verb+(data . 1234567890)+ is dumped as a bignum.

\item
The ``codevector'' tag means that the following vector will have to be dumped
as a byte vector. This tag is necessary only because there in not a standard
print syntax for byte vectors; if there were one, we could use the ``data''
tag.

\item
Each ``constantvector'' will be recursively traversed to check for tags.

\item
A ``global'' forces the heap dumper to insert a reference to the global
variable cell for the given global variable; the datum in this case is
a symbol naming the global variable.

\item
If the tag is ``bits'', then the datum (an integer) is dumped as is without 
any change of tags. This is useful internally in the heap dumper.

\end{itemize}

\section{Heap Format}

This is the format for the current (version 1) heap, which is a single memory
area heap -- there are no provisions for the static area (see the GC section,
below).

\begin{itemize}
\item Heap proper is always padded to 8-byte boundary
\item Version number (1 word)
\item Roots (38 words in current version; number of roots and exact layout
\item depends on the version. The roots and the order is typically the same
as defined in ``offsets.h''.)
\item Heap length in words (1 word)
\item Heap itself (n words)
\end{itemize}

\end{document}
