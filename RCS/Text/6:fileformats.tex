% -*-TeX -*-
% $Id: fileformats.tex,v 1.1 92/02/18 11:52:51 lth Exp Locker: lth $

% -*- LaTeX -*-

\documentstyle[10pt]{article}
\newcommand{\reg}[1]{{\sc \%#1}}

\topmargin      -2.0cm
\oddsidemargin   0.0cm
\evensidemargin  0.0cm
\textwidth      6.5in
\textheight     9.5in
\parindent       0.0cm
\parskip         0.4cm


\title{Larceny Note \#6: File Formats}
\author{Lars Thomas Hansen}

\begin{document}
\maketitle

\section{Introduction}

We operate with several kinds of files:

\begin{itemize}
\item 
Scheme source code comes in text files with an extension of ``.sch'' or
``.scm''; the latter is archaic and should be done away with. These files
are input to the compiler and can be loaded interactively with the ``load''
procedure when the interactive environment becomes operational.

\item
MacScheme assembly language code comes in two flavors: that which may be
directly assembled (encoded format) and that which must be processed before
being assembled (unencoded format). The former files have an extension ``.lap''
(for ``List Assembly Program''). The latter have an extension ``.mal'' (for
``MacScheme Assembly Language''). Both can be processed by the assembler,
but the contents of ``.mal'' files are passed through ``eval'' before being
assembled; see below for details.

\item
SPARC object code comes in two flavors: that which may be used to build an
initial heap image and that which may be loaded interactively. The former
files have a ``.lop'' (for ``Lisp Object Program'') extension; the latter
have a ``.fasl'' extension (for ``Fastload'').

\item
Heap files all have a ``.heap'' extension and are distinguished by their
internal organization. There are several kinds of heaps: those which are
loaded into the tenured area; those which are loaded into the ephemeral
area; and those which are loaded into several areas, including the static
area. Not all heap images are useful in Larceny proper, but are used for
research purposes.

\end{itemize}

The format of scheme source files is obvious; the following sections
describe the other file formats in detail.

\section{MacScheme Assembly Language Files}

Files with MacScheme assembly language consists of lists of assembly language
instructions. Each list is interpreted as the instruction list (code list)
of a {\tt lambda} instruction at the top level; the environment field of the
resulting procedure is void.

Encoded (``.lap'') files are raw lists of instructions where the opcode is a
small nonnegative integer corresponding to the real opcode. The exact value
of the code for any particular opcode is defined in a table within the
compiler; when this table changes, all ``.lap'' files have to be remade from
sources.

Unencoded (``.mal'') files are scheme expressions which evaluate to raw
lists of instructions where the opcode is a small nonnegative integer
corresponding to the real opcode. Typically this is accomplished with some
quasiquoted list where the names of the opcoded are the unquoted names of
the MacScheme instruction prefixed by ``\$'':

\begin{minipage}{\linewidth}
\begin{verbatim}
    `((,$lambda ((,$.proc)
                 (,$args= 2)
                 (,$reg 1))
                0
                (#t))
      (,$setglbl foo)
      (,$return))
\end{verbatim}
\end{minipage}

\section{SPARC Object Code Files}

Fastload (``.fasl'') files contain executable Scheme expressions in source
form; extra magic syntax is used to represent literal procedures, global
variable references, and code vectors. {\em This format is temporary.} Using
this format, the current top level can load binary files into the running
system without having to use the compiler or assembler.

A procedure is represented using the magic syntax \verb+#^P( ... )+ (where
\verb+^P+ stands for the ASCII character Control-P); the items in the
parentheses are the contents of the procedure's data slots.

A code vector is a byte vector; it is represented using the magic syntax
\verb+#^B" ... "+, where the items in the double quotes are extended (8-bit)
ASCII characters; double quotes and backslashes are escaped as usual.

A reference to a global variable is represented as \verb+#^G<symbol>+; the
symbol names the global procedure. A real reference to the global value cell
is inserted at the point where this reference appears.

Raw object (``.lop'') files consists of segments. A segment is a pair:

\begin{itemize}
\item
The car of the segment is a byte vector (externally represented as a vector
in some host implementations) containing the raw bytes of the
position-independent SPARC object code. This code is fully compiled and will
not need to be further patched at load time. The bytes are laid out in order
of increasing addresses for fast loading. The byte vector is exactly long
enough to hold the object code; no padding is added.

\item
The cdr of the segment is a vector containing all non-inlined constants
used by the procedure. Constants may share structure if the assembler
determined that they could. Examples of constants include lists, flonums, 
and code and constant vectors for nested procedures which are to be created
at run time. Each entry in the constant vector is a list. The car of the
pair is the tag of the entry; the cadr is the actual data. There are five
kinds of tags: ``data'', ``codevector'', ``constantvector'', ``global'',
and ``bits''. 
\end{itemize}

The tags and their data are interpreted this way:

\begin{itemize}
\item
The ``data'' tag is used for raw data which is to be dumped
into the heap as is, but with the correct tags added. For example,
\verb+(data 1234567890)+ is dumped as a bignum.

\item
The ``codevector'' tag means that the following vector will have to be dumped
as a byte vector. This tag is necessary only because there is not a standard
print syntax for byte vectors; if there were one, we could use the ``data''
tag.

\item
Each ``constantvector'' will be recursively traversed to check for tags.

\item
A ``global'' forces the heap dumper to insert a reference to the global
variable cell for the given global variable; the datum in this case is
a symbol naming the global variable.

\item
If the tag is ``bits'', then the datum (an integer) is dumped as is without 
any change of tags. This is useful internally in the heap dumper, but there is
no reason why this tag could not exist in a ``.lop'' file as well.

\end{itemize}

\section{Larceny Heap Files}

Heaps are distinguished by their major and minor version numbers. The major
version number distinguishes heap types; the minor version number
distinguishes variations within a major type.

There are two major types, known as ``unified'' and ``split''. Type 0 is the
unified heap. This heap type contains only one data area; it is loaded into
the tenured or ephemeral area (based on a command line switch). Type 1 is
the split heap. This heap type contains two data areas: one of is loaded
into the static area, the other is loaded into the tenured or ephemeral area
(again based on a command line switch).

Heap data areas are always padded to 8 bytes, and words are stored
big-endian.  The first word of the heap is the version number; the high 16
bits have the major number and the low 16 bits have the minor number.
Following the version number is the set of roots; these values will be the
initial value of the system roots. The exact number of the roots and their
ordering varies between minor versions, but it is the same for all major
versions with the same minor version. The order of the roots is not
described here. Following the roots is information which depends on the
major version number.

All addresses in the heap are given relative to a base address of 0; the heap
loader adjusts the base addresses at load time to fit the actual address of
the heap. This is easily done since all words in the heap are tagged (or if
they are not, they are preceded by a header which identifies them as untagged).
Also see below.

For heap type 0, the roots are followed by a word which gives the heap
size in words, and then the heap data itself.

For heap type 1, the roots are followed by a word which gives the size of
the static heap, then a word which gives the size of the tenured/ephemeral
heap, then the data for the static heap, and finally the data for the
tenured/ephemeral heap.  Both data areas are padded to 8 bytes. Since the
tenured/ephemeral area may have references into the static area, these
references must be recognized and resolved specially. The convention is that
every pointer in the tenured/ephemeral heap image which points into the
static are has the high bit set. This tells the heap loader to resolve the
address relative to the base of the static area rather than relative to the
base of the tenured/ephemeral area.

\end{document}
